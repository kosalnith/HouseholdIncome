\begin{table}[H]
	\centering
	\caption{Summary statistics for macroeconomic variables}
	\label{app:tabm1}
	\resizebox{.95\textwidth}{!}{%
		\begin{tabular}{@{}lccccccc@{}} 
			\hline \hline
			&       Obs.&        Mean&          S.D.&         Min.&         Max. & Unit& Source\\
			\hline
			Inflation         &          46&        3.01&        1.51&        0.70&        7.06 & Percent& NIS\\
			Output      &          46&        4.83&        1.39&        2.78&        7.45 & Billion dollars &NIS\\
			Unemployment rates       &          46&        0.43&        0.23&        0.13&        0.77 & Percent& ILO\\
			Interest rates       &          46&        5.26&        0.77&        4.19&        6.91 &Weighted average rate&NBC\\
			Exchange rates     &          46&     4071.58&       52.64&     3985.60&     4257.36 &Khmer riels/US dollars& NBC\\
			M2         &          46&       15.66&        9.84&        4.13&       36.29 &Billion dollars&NBC\\

			\hline\hline
		\end{tabular}%
	}
	\begin{tablenotes}
	\footnotesize
	\item \textbf{Note:} This table displays the descriptive statistics of macroeconomic variables that using for the investigation of monetary policy shocks. The second column reports the number of observations, the third presents the mean, and the fourth column is standard deviation errors. Many variables are observed quarterly, and some variables, such as the unemployment rate and output, are recorded annually. I suppose the real input growth and the employment rate have the same proposition every quarter in a whole year, and then I multiply it with 4 to get the quarter data. The dataset comes from the National Institute of Statistics, the International Labor Organization, and the National Bank of Cambodia.       

	\end{tablenotes} 
	
\end{table}