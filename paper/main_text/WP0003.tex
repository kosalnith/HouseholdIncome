\documentclass[11pt,letterpaper]{article}
\usepackage[utf8]{inputenc}
\usepackage[sc]{mathpazo}

\usepackage[T1]{fontenc}
\usepackage{amsmath}
\usepackage{amsfonts}
\usepackage{amssymb}
\usepackage{graphicx}
\usepackage{bbm}
\usepackage{hyperref}
\usepackage{natbib}
\usepackage[width=16.00cm, height=22.00cm]{geometry}
\usepackage{tablefootnote}
\usepackage{multirow}
\usepackage{longtable}
\usepackage{rotating, tabularx}
\usepackage[table,xcdraw]{xcolor}
\usepackage[dvips]{xcolor}
\usepackage{pdflscape, lipsum}
\usepackage{tabularx}
\usepackage{url}
\usepackage{float}
\usepackage[capposition=top]{floatrow}


\definecolorseries

\usepackage[pagewise]{lineno}
%\linenumbers

\renewcommand{\baselinestretch}{1.3}
%\linespread{1.3}         % Palladio needs more leading (space between lines)


\hypersetup{colorlinks,linkcolor={[cmyk].75,1,0,0},citecolor={[cmyk]0,0.88502,1,0}, filecolor={violet}, urlcolor={[cmyk]0,1,.7,0}}  
%magenta
%\hypersetup{colorlinks=false, urlbordercolor=blue}

\usepackage{booktabs,siunitx,amsmath,caption}

%-------------------------------------------------------------------
\author{Nith Kosal\footnote{Correspondence: \href{mailto:nithkosal@futureforum.asia}{\texttt{nithkosal@futureforum.asia}}. Webpage: \url{https://nithkosal.github.io}.} \\
	\small{Future Forum}  
	\and
	Phay Thounimith\footnote{Correspondence: \href{mailto:phaythounimith96@gmail.com }{\texttt{phaythounimith96@gmail.com}}.}  \\
	\small{Royal School of Administration}  
}


\title{Monetary Policy and Household Income Distribution}
%-------------------------------------------------------------------
\begin{document}
	
	
	 %PRELIMINARY AND INCOMPLETE 
	 
%-------------------------------------------------------------------	  
\maketitle
	\begin{abstract}
		    
	\end{abstract}
	\textit{\textbf{JEL Classification:}} \\
	\textit{\textbf{Keywords:} Monetary Policy, Income Inequality, Structural Change}
	
	\clearpage
%-----------------------------------------------------------------------------------
\begin{quotation}
	The quotation about this paper within a citation. 
\end{quotation}
%-----------------------------------------------------------------------------------
\section{Introduction}\label{sec:intro}

In common theoretical economy has showed that the monetary policy is the process by which the monetary administration of a country, like the central banks or currency board, controls the supply of money, usually targeting an inflation rate or interest rate to ensure price stability and general trust in the currency. Further goals of a monetary policy are usually to contribute to economic growth and stability, to lower unemployment, and to maintain predictable exchange rates with other currencies.
 
The Solow model is therefore assuming that the economy as a whole in particular households and firms consumes a constant friction of its output each period. Even so, the assumption of a constant saving rate is generally not, because it is going to be optimal from a microeconomic perspective in the short run.  


%-----------------------------------------------------------------------------------

\textbf{Related Literature. }
%-----------------------------------------------------------------------------------

Since the end of the financial crisis in 2009, currently the community transmission of COVID-19 has been a major challenge for Cambodia's healthcare system and household incomes. According to the \cite{Karamba2021}, a fifth-round result of the World Bank's High-Frequency Phone Survey of Households in Cambodia found that 45 percent of households experienced declines in income in December 2020 and March 2021. This compares to 48 percent in Round 4 that experienced declines between October and December, 51 percent in Round 3 that experienced declines between August and October. Where in Round 2, households income report declines 63 percent between May and August 2020 and 83 percent in Round 1 that experienced declines between the first COVID-19 outbreak and May 2020. Despite the fact that Phnom Penh and its nearby town were put in lockdown, but we does not yet any report to show about household income lost during the period of lockdown and travail restrictions.  
%-----------------------------------------------------------------

\section{Background}\label{sec:back}
%----------------------------------------------------------------
Much of the attempt from  central bank of Cambodia is to provide the stability macroeconomics policy, in particular the stable monetary policy, within the US and Riel exchange rate around 4000 Riels per \$1 and the inflation rate around 3 percent per year. For more than four decades, the exchange rate stability is the main objective the central bank of Cambodia play around within the monetary policy.  
\section{Data Sources and Descriptive Statistics}\label{sec:data}
%-----------------------------------------------------------------
\subsection{Sources, Variables and Sample}	
Our analysis focus sorely Cambodia over the period 2008Q1-2020Q4. In fact, surveys have been used by several studies on the im- pact of monetary policy on income inequality (e.g., Coibion et al. (2017) for the U.S.; Mumtaz and Theophilopoulou (2017) for the U.K.; Inui et al. (2017) for Japan; and Casiraghi et al. (2018) for Italy).
%-----------------------------------------------------------------------------------
\subsection{Percentiles and the Inequality Index}	
%-----------------------------------------------------------------------------------

\section{A Model of Monetary Policy, Inequality and Structural Change}\label{sec:model}

\subsection{Interest Rates Exposure}
To analysis the influence  of monetary policy (MP) in Cambodia, we follow Romer and Romer (2004) to identify innovations to monetary policy purged of anticipatory effects associated with economic conditions. 
\begin{equation}\begin{split}
\Delta nbcr_{m} = \alpha + \beta nbcrb_{m} + \sum_{i = -1}^{2}{\gamma \widetilde{\Delta y}_{mi}} + \sum_{i = -1}^{2} \lambda_{i} (\widetilde{\Delta y}_{mi} - \widetilde{\Delta y}_{m-1, i}) + \sum_{i = -1}^{2} \varphi_{i}\widetilde{\pi}_{mi} \\
+ \sum_{i = -1}^{2}\theta_{i}(\widetilde\pi_{mi} - \widetilde\pi_{m-1, i}) + \rho \widetilde{u}_{m0} + \varepsilon_{m}
\end{split}\end{equation}

$\Delta nbcr_{m}$ denotes the change in the intended interest rate of the central bank around the innovation policy with the new implementation $m$. $nbcrb_{m}$ is the level of the intended interest rate before any changes associated with any new regulations $m$. $\widetilde{\pi}$, $\widetilde{\Delta y}$, and $\widetilde{u}$ refer to the forecasts of inflation, real out growth, and the unemployment rate. The $i$ subscripts refer to the horizon of the forecast: $-1$ is the previous quarter, $0$ is the current quarter, while $1$ and $2$ are one and two quarters ahead, respectively. The forecast for the previous quarter is often actual data rather than a forecast.  

%------------------------------------------------------------------
\subsection{Inflation Tax}
%------------------------------------------------------------------
\subsection{Income Composition}
%------------------------------------------------------------------
\begin{equation}}
	\frac{Y_{j;t-n}-\overline{Y}_{j;<t}}{\overline{D}_{j;<t}} = \sum_{k=1}^{K} \mathbbm{1}[j\in k]\begin{bmatrix}\alpha^{k} +\beta^{k}(-\Delta i_{t}) + \delta^{k} Z_{t}\end{bmatrix} + \varepsilon_{j; t}
\end{equation}

In the economy, households' choices about to save and borrow in financial markets are crucial to their consumption decisions. 
%------------------------------------------------------------------

\subsection{Savings Redistribution}
%------------------------------------------------------------------
\subsection{Earning Heterogeneity}
%------------------------------------------------------------------
\subsection{Fiscal Policy} 
%------------------------------------------------------------------
\subsection{Technological Change and Innovation}
%------------------------------------------------------------------
\subsection{Environmental Change}
%------------------------------------------------------------------
\subsection{Economics of Crises}
%------------------------------------------------------------------

\section{Empirical Results}\label{sec:results}

%-------------------------------------------------------------------


%-----------------------------------------------------------------------------------
\section{Discussion and Robustness}
%-----------------------------------------------------------------------------------
  
\section{Conclusion}\label{sec:conclusion}
%-----------------------------------------------------------------------------------

%-------------------------------------------------------------------
\clearpage	
{\small
	\bibliographystyle{ecca}
	\bibliography{References}
}	
%--------------------------------------------------------------------
\clearpage

\appendix 
	\part*{Appendix}\label{app:dix}
	\pagenumbering{Arabic}
	\renewcommand*{\thepage}{A\arabic{page}}
	\numberwithin{equation}{section}
	\renewcommand{\thetable}{A\arabic{table}}
	%\renewcommand{\thefigure}{A\arabic{figure}}	
	
\end{document}
