\documentclass[11pt,letterpaper]{article}
\usepackage[utf8]{inputenc}
\usepackage[sc]{mathpazo}

\usepackage[T1]{fontenc}
\usepackage{amsmath}
\usepackage{amsfonts}
\usepackage{amssymb}
\usepackage{graphicx}
\usepackage{bbm}
\usepackage{hyperref}
\usepackage{apacite}
\usepackage{natbib}
\usepackage{har2nat}
\usepackage[width=16.00cm, height=22.00cm]{geometry}
\usepackage{tablefootnote}
\usepackage{multirow}
\usepackage{longtable}
\usepackage{rotating, tabularx}
\usepackage[table,xcdraw]{xcolor}
\usepackage[dvips]{xcolor}
\usepackage{pdflscape, lipsum}
\usepackage{tabularx}
\usepackage{url}
\usepackage{float}
\usepackage[capposition=top]{floatrow}


\definecolorseries

\usepackage[pagewise]{lineno}
%\linenumbers

\renewcommand{\baselinestretch}{1.3}
%\linespread{1.3}         % Palladio needs more leading (space between lines)


\hypersetup{colorlinks,linkcolor={[rgb]0.7, 0.11, 0.11},citecolor={[rgb]0.7, 0.11, 0.11}, filecolor={[rgb]0.7, 0.11, 0.11}, urlcolor={[rgb]0.7, 0.11, 0.11}}  
%magenta
%\hypersetup{colorlinks=false, urlbordercolor=blue}
%0,0.88502,1,0
\usepackage{booktabs,siunitx,amsmath,caption}

%-------------------------------------------------------------------
\author{Nith Kosal\footnote{Correspondence: \href{mailto:nithkosal@futureforum.asia}{\texttt{nithkosal@futureforum.asia}}. Webpage: \url{https://nithkosal.github.io}.} \\
	\small{Future Forum}  
	\and
	Phay Thounimith\footnote{Correspondence: \href{mailto:phaythounimith96@gmail.com }{\texttt{phaythounimith96@gmail.com}}.}  \\
	\small{Royal School of Administration}  
}


\title{Monetary Policy and Household Income Distribution}
%-------------------------------------------------------------------
\begin{document}
	

	 %PRELIMINARY AND INCOMPLETE 
	 
%-------------------------------------------------------------------	  
\maketitle
	\begin{abstract}
		    
	\end{abstract}
	\textit{\textbf{JEL Classification:}} \\
	\textit{\textbf{Keywords:} Monetary Policy, Income Inequality, Structural Change}
	
	\clearpage
%-----------------------------------------------------------------------------------
\begin{quotation}
	The quotation about this paper within a citation. 
\end{quotation}
%-----------------------------------------------------------------------------------
\section{Introduction}\label{sec:intro}

In common theoretical economy has showed that the monetary policy is the process by which the monetary administration of a country, like the central banks or currency board, controls the supply of money, usually targeting an inflation rate or interest rate to ensure price stability and general trust in the currency. Further goals of a monetary policy are usually to contribute to economic growth and stability, to lower unemployment, and to maintain predictable exchange rates with other currencies.
 
The Solow model is therefore assuming that the economy as a whole in particular households and firms consumes a constant friction of its output each period. Even so, the assumption of a constant saving rate is generally not, because it is going to be optimal from a microeconomic perspective in the short run.  

	\citet{Albanesi2001} \cite{Wahid2019}
%-----------------------------------------------------------------------------------

\textbf{Related Literature.} Our work contributes to a growing literature studying the distributional effects on monetary policy transmission on inequality. \citep{Coibion2017} used micro-level data from the Consumption Expenditures Survey to study the effects of monetary policy shocks to consumption and income inequality in the United States in the period of 1980.  
%-----------------------------------------------------------------------------------


%-----------------------------------------------------------------
\clearpage
\section{Background}\label{sec:back}
%----------------------------------------------------------------
Given the importance of the institutional framework for our identification strategy, we describe the background in some detail in this section.
\subsection{Monetary Policy}
%-----------------------------------------------------------------------------------
Economic and political system of capitalism throughout Cambodia was collapsed and moved into the communist states during the period of the Khmer Rouge in 1975-1979. At the time, the Cambodian riel banknote and banking system was completely abolished and destroyed. The country was run without a monetary policy and money system \cite{Duma2014}. About 9 months after the fall of the Khmer Rouge regime on 7 January 1979, the central bank was reestablished on 10 October 1979 and first reissued the riel banknotes on 20 March 1980 \cite{Visoth2010}. The riel currency can not applied for all domestic transactions in the 1980s due to there was limited the monetization of economy. The country was put into monetary plurality system with many foreign currencies are freely used in their territory and most non-plan translations were based on barter and gold. People living along the border with Thailand and Vietnam experienced use Thai baht and Vietnamese dong for the mean of payment. 

It is important to remember the monetary crisis during 1989-1992, when the collapse of Eastern Europe and the end of financial support from the former Soviet Union, led to the consequence that high inflation rate from about 90 to 177 percent per annual \cite{Chhun2005}. This is showed that Cambodia still lacked independence in formulating and conducting monetary policy, especially in relation to financing of the budget and uncertainty at the political level. Interestingly, the annually inflation rate down rapidly to 31 percent in the election year of 1993 and to 24 percent in 1994. This changed due to the Paris Peace Agreements on 23 October 1991 was a political detour of the transforming from planning economy to free market economy.\footnote{In 1989, Cambodia transformed a mono-banking system into a two-tier banking system. The first commercial bank was established in 1991 under the form of state joint venture bank for attracting investors and serving banking activity doing the United National Transitional Authorities being present in Cambodia (UNTAC).} With the new government in the mid 1994, the central bank with the support from the International Monetary Fund to reform policy framework focus on financial and technical assistance to develop macroeconomics policy.

For more than two decades, the exchange rate stability and low inflation rate are the principle mission of the National Bank of Cambodia is to determine and direct monetary policy instruments aiming to facilitate economic development and sustainable macroeconomic environment. Dollarization could bring the stability of exchange rate and inflation but it also may generate difficulties for the central banks in running an effective MP since can control only part of the total broad money \cite{Mohsin1992}.\footnote{High dollarization compared other developing economies \cite{Ra2008}, for example, compared to it neighboring country, Vietnam and Laos \cite{Khou2013}} This institutional fact is a key part of our identification strategy. Theory tells us that to keep the exchange rate fixed in an open economy, the central bank must use the policy rate to control the demand for local currency and therefore cannot at the same time use it to control other local economic conditions \cite{Fleming1962, Mundell1963}. Although there is some alignment between business cycles in Cambodia, the currency peg introduces a source of exogenous variation in Cambodian monetary policy that we will exploit in the empirical analysis. In general, the NBC has set the official exchange rate with the difference of one percent as compared to the market rate. The exchange rate Khmer riel against US dollar was set aound 4,000 riels per \$1. The exchange rate of Vietnam dong against Khmer riel and Thai baht against Khmer riel are associated with bilateral exchange rate of US dollar against Vietnam dong and US dollar against Thai baht. 


Local currency is estimated as just 10 percent of the total board money. In \citet{Khou2013} showed that circulation of the riel has grown since the early 2000s, in particular, in the second half. Between 1989 and 2000, the amount of riels in circulation varied slightly and was around 100 million dollars, while the economic growth rate exceeded 5 percent on average. In contrast, it increased by about 20 percent between 2001 and 2010. This allowed the volume of the banknotes in circulation to pass from about 100 million dollars in the period of 1989-2000 to nearly 800 million dollars in 2010. Yet bank deposits in riel remain very low; they represent less than 5 percent of the total deposits. 

\\






\\

The reserve requirement is the main Cambodia's MP to control the prudential and liquidity management. As the time of the pandemic, many economic activities was hard hit in demand and supply shock, to facilitate sable money supply in the market, the MP Committee decuted  to reduce the reserve requirement for both local currency and foreign currency US dollar to 7 percent.\footnote{The decision to counties the rate of reserve requirement of 7 percent was introduced since 2021. The summary result of the 56 Monetary Policy Committee Meeting was released in the official Facebook page of NBC on 23 August 2021. The settlement can be found via this link: \url{https://www.facebook.com/nationalbankofcambodiaofficial/photos/3897621040348119}} 

\\


\\



%Much of the attempt from  central bank of Cambodia is to provide the stability macroeconomics policy, in particular the stable monetary policy, within the US and Riel exchange rate around 4000 Riels per \$1 and the inflation rate around 3 percent per year. 

%-----------------------------------------------------------------------------------
\subsection{Household Income}
%-----------------------------------------------------------------------------------
In 2020, the total number of Cambodian normal households is about 3.6 million with 15.9 million people living in Cambodia \cite{NIS2020}. Gross Domestic Product (GDP) per Capita has increased from \$1,555 in 2018 to \$1,694 in 2019. 

The total monthly income of Cambodian households is estimated to be \$567 in 2019/20, which is an increase by 16 percent if compared to 2017 and it is increased by 58 percent if compared to year 2014. Cambodian household’s disposable income in 2019/20 increased by 16 percent if compared to 2017 and increased by 59 percent if compared to 2014. The higher increase compared to the increase of total income is mainly due to the data collection on current transfers paid changed from diary methods to recall method since 2012.


The central bank was first introduced an 18 percent cap on annual interest rate of microfinance institute loans both denominated in local and US dollar currency. Even through not much prevent literature to study the inference of this interest rate ceiling on household. But we can assume that it will direct benefit to family financing.\footnote{For example, in \citet{Heng2021}, they studied the effect of the interest rate cap of MFIs and financial inclusion by using the dataset on financial institution and aggregate level from 2016-2019. They result found}  

Since the end of the financial crisis in 2009, currently the community transmission of COVID-19 has been a major challenge for Cambodia's healthcare system and household incomes. According to the \cite{Karamba2021}, a fifth-round result of the World Bank's High-Frequency Phone Survey of Households in Cambodia found that 45 percent of households experienced declines in income in December 2020 and March 2021. This compares to 48 percent in Round 4 that experienced declines between October and December, 51 percent in Round 3 that experienced declines between August and October. Where in Round 2, households income report declines 63 percent between May and August 2020 and 83 percent in Round 1 that experienced declines between the first COVID-19 outbreak and May 2020. Despite the fact that Phnom Penh and its nearby town were put in lockdown, but we does not yet any report to show about household income lost during the period of lockdown and travail restrictions.  
%-----------------------------------------------------------------------------------
Many decades after the fall of the Khmer Rouge, now Cambodia has moved to new transposition with a lower middle-income status since 2015. 
\section{A Model of Monetary Policy Transmission Channel}\label{sec:model}

%-----------------------------------------------------------------------------------
\subsection{Aggregate Income Channel}
%-----------------------------------------------------------------------------------

\subsection{Household}
%-----------------------------------------------------------------------------------

\subsection{Interest Rate Exposure Channel}
To analysis the influence  of monetary policy (MP) in Cambodia, we follow Romer and Romer (2004) to identify innovations to monetary policy purged of anticipatory effects associated with economic conditions. 
\begin{equation}\begin{split}
\Delta rrr_{m} = \alpha + \beta rrrb_{m} + \sum_{i = -1}^{2}{\gamma_{i} \widetilde{\Delta y}_{mi}} + \sum_{i = -1}^{2} \lambda_{i} (\widetilde{\Delta y}_{mi} - \widetilde{\Delta y}_{m-1, i}) + \sum_{i = -1}^{2} \varphi_{i}\widetilde{\pi}_{mi} \\
+ \sum_{i = -1}^{2}\theta_{i}(\widetilde\pi_{mi} - \widetilde\pi_{m-1, i}) + \rho \widetilde{u}_{m0} + \varepsilon_{m}
\end{split}\end{equation}

\noindent where $m$ dotes the MP Committee meeting, $\Delta rrr_{m}$ is the change in the intended reserve requirement rate of the central bank around the innovation policy with the new implementation $m$. $nbcrb_{m}$ is the level of the intended interest rate before any changes associated with any new regulations $m$. $\widetilde{\pi}$, $\widetilde{\Delta y}$, and $\widetilde{u}$ refer to the forecasts of inflation, real out growth, and the unemployment rate. The $i$ subscripts refer to the horizon of the forecast: $-1$ is the previous quarter, $0$ is the current quarter, while $1$ and $2$ are one and two quarters ahead, respectively. The forecast for the previous quarter is often actual data rather than a forecast.  

%------------------------------------------------------------------
\subsection{Inflation Tax}
%------------------------------------------------------------------
\subsection{Income Composition}
%------------------------------------------------------------------
sß\begin{equation}}
	\frac{Y_{j;t-n}-\overline{Y}_{j;<t}}{\overline{D}_{j;<t}} = \sum_{k=1}^{K} \mathbbm{1}[j\in k]\begin{bmatrix}\alpha^{k} +\beta^{k}(-\Delta i_{t}) + \delta^{k} Z_{t}\end{bmatrix} + \varepsilon_{j; t}
\end{equation}

In the economy, households' choices about to save and borrow in financial markets are crucial to their consumption decisions. 
%------------------------------------------------------------------

\subsection{Savings Redistribution}
%------------------------------------------------------------------
\subsection{Earning Heterogeneity}
%------------------------------------------------------------------
\section{Data Sources and Descriptive Statistics}\label{sec:data}
%-----------------------------------------------------------------
\subsection{Sources, Variables and Sample}	
Our analysis focus sorely Cambodia over the period 2008Q1-2020Q4. In fact, surveys have been used by several studies on the im- pact of monetary policy on income inequality (e.g., Coibion et al. (2017) for the U.S.; Mumtaz and Theophilopoulou (2017) for the U.K.; Inui et al. (2017) for Japan; and Casiraghi et al. (2018) for Italy).
%-----------------------------------------------------------------------------------
\subsection{Percentiles and the Inequality Index}	
%-----------------------------------------------------------------------------------


\section{Empirical Results}\label{sec:results}

%-------------------------------------------------------------------

\section{Robustness Checks}\label{sec:robust}

%-------------------------------------------------------------------

%-----------------------------------------------------------------------------------
\section{Discussion}
%-----------------------------------------------------------------------------------
  
\section{Conclusion}\label{sec:conclusion}
%-----------------------------------------------------------------------------------

%-------------------------------------------------------------------
\clearpage	
{\small
	%\bibliographystyle{jpe} % is refer to the Journal of Political Economy
	%\bibliographystyle{apecon}
	%\bibliographystyle{ier}
	\bibliographystyle{aea}  % is refer to the American Economic Association
	%\bibliographystyle{ecca} % is refer to the Journal of Economica
	\bibliography{References}
}	
%--------------------------------------------------------------------
\clearpage

\appendix 
	\part*{Appendix}\label{app:dix}
	\pagenumbering{Arabic}
	\renewcommand*{\thepage}{A\arabic{page}}
	\numberwithin{equation}{section}
	\renewcommand{\thetable}{A\arabic{table}}
	%\renewcommand{\thefigure}{A\arabic{figure}}	
	
\end{document}
