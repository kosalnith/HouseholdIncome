\documentclass[11pt,letterpaper]{article}
\usepackage[utf8]{inputenc}
\usepackage[sc]{mathpazo}

\usepackage[T1]{fontenc}
\usepackage{amsmath}
\usepackage{amsfonts}
\usepackage{amssymb}
\usepackage{graphicx}
\usepackage{comment}
\usepackage{bbm}
\usepackage{hyperref}
\usepackage{apacite}
\usepackage{natbib}
\usepackage{har2nat}
\usepackage[width=16.00cm, height=22.00cm]{geometry}
\usepackage{tablefootnote}
\usepackage{multirow}
\usepackage{longtable}
\usepackage{rotating, tabularx}
\usepackage{booktabs}
\usepackage{threeparttable}
\usepackage[table,xcdraw]{xcolor}
\usepackage[dvips]{xcolor}
\usepackage{pdflscape, lipsum}
\usepackage{tabularx}
\usepackage{url}
\usepackage{float}
\usepackage[capposition=top]{floatrow}


\definecolorseries

\usepackage[pagewise]{lineno}
%\linenumbers

\renewcommand{\baselinestretch}{1.3}
%\linespread{1.3}         % Palladio needs more leading (space between lines)


\hypersetup{colorlinks,linkcolor={[rgb]0.7, 0.11, 0.11},citecolor={[rgb]0.7, 0.11, 0.11}, filecolor={[rgb]0.7, 0.11, 0.11}, urlcolor={[rgb]0.7, 0.11, 0.11}}  
%magenta
%\hypersetup{colorlinks=false, urlbordercolor=blue}
%0,0.88502,1,0
\usepackage{booktabs,siunitx,amsmath,caption}

%-------------------------------------------------------------------
\author{Nith Kosal\footnote{Correspondence: \href{mailto:nithkosal@futureforum.asia}{\texttt{nithkosal@futureforum.asia}}. Webpage: \url{https://nithkosal.github.io}.} \\
	\small{Future Forum}  
	\and
	Phay Thounimith\footnote{Correspondence: \href{mailto:phaythounimith96@gmail.com }{\texttt{phaythounimith96@gmail.com}}.}  \\
	\small{Royal School of Administration}  
}


\title{Monetary Policy and Household Income Distribution}
%-------------------------------------------------------------------
\begin{document}
	

	 %PRELIMINARY AND INCOMPLETE 
	 
%-------------------------------------------------------------------	  
\maketitle
	\begin{abstract}
		    
	\end{abstract}
	\textit{\textbf{JEL Classification:}} \\
	\textit{\textbf{Keywords:} Monetary Policy, Income Inequality, Structural Change}
	
	\clearpage
%-----------------------------------------------------------------------------------
\begin{quotation}
	The quotation about this paper within a citation. 
\end{quotation}
%-----------------------------------------------------------------------------------
\section{Introduction}\label{sec:intro}

In common theoretical economy has showed that the monetary policy is the process by which the monetary administration of a country, like the central banks or currency board, controls the supply of money, usually targeting an inflation rate or interest rate to ensure price stability and general trust in the currency. Further goals of a monetary policy are usually to contribute to economic growth and stability, to lower unemployment, and to maintain predictable exchange rates with other currencies.
 
The Solow model is therefore assuming that the economy as a whole in particular households and firms consumes a constant friction of its output each period. Even so, the assumption of a constant saving rate is generally not, because it is going to be optimal from a microeconomic perspective in the short run.  

%-----------------------------------------------------------------------------------

\textbf{Related Literature.} Our work contributes to a growing literature studying the distributional effects on monetary policy transmission on inequality. \citep{Coibion2017} used micro-level data from the Consumption Expenditures Survey to study the effects of monetary policy shocks to consumption and income inequality in the United States in the period of 1980.  
%-----------------------------------------------------------------------------------


%-----------------------------------------------------------------
\clearpage
\section{Background}\label{sec:back}
%----------------------------------------------------------------
Given the importance of the institutional framework for our identification strategy, we describe the background in some detail in this section.
\subsection{Monetary Policy}
%-----------------------------------------------------------------------------------
Economic and political system of capitalism throughout Cambodia was collapsed and moved into the communist states during the period of the Khmer Rouge in 1975-1979. At the time, the Cambodian riel banknote and banking system was completely abolished and destroyed. The country was run without a monetary policy and money system \cite{Duma2014}. About 9 months after the fall of the Khmer Rouge regime on 7 January 1979, the central bank was reestablished on 10 October 1979 and first reissued the riel banknotes on 20 March 1980 \cite{Visoth2010}. The riel currency can not applied for all domestic transactions in the 1980s due to there was limited the monetization of economy. The country was put into monetary plurality system with many foreign currencies are freely used in their territory and most non-plan translations were based on barter and gold. People living along the border with Thailand and Vietnam experienced use Thai baht and Vietnamese dong for the mean of payment. 

It is important to remember the monetary crisis during 1989-1992, when the collapse of Eastern Europe and the end of financial support from the former Soviet Union, led to the consequence that high inflation rate from about 90 to 177 percent per annual \cite{Chhun2005}. This is showed that Cambodia still lacked independence in formulating and conducting monetary policy, especially in relation to financing of the budget and uncertainty at the political level. Interestingly, the annually inflation rate down rapidly to 31 percent in the election year of 1993 and to 24 percent in 1994. This changed due to the Paris Peace Agreements on 23 October 1991 was a political detour of the transforming from planning economy to free market economy.\footnote{In 1989, Cambodia transformed a mono-banking system into a two-tier banking system. The first commercial bank was established in 1991 under the form of state joint venture bank for attracting investors and serving banking activity doing the United National Transitional Authorities being present in Cambodia (UNTAC).} With the new government in the mid 1994, the central bank with the support from the International Monetary Fund to reform policy framework focus on financial and technical assistance to develop macroeconomics policy.

For more than two decades, the exchange rate stability and low inflation rate are the principle mission of the National Bank of Cambodia is to determine and direct monetary policy instruments aiming to facilitate economic development and sustainable macroeconomic environment. Dollarization could bring the stability of exchange rate and inflation but it also may generate difficulties for the central banks in running an effective MP since can control only part of the total broad money \cite{Mohsin1992}.\footnote{High dollarization compared other developing economies \cite{Ra2008}, for example, compared to it neighboring country, Vietnam and Laos \cite{Khou2013}} This institutional fact is a key part of our identification strategy. Theory tells us that to keep the exchange rate fixed in an open economy, the central bank must use the policy rate to control the demand for local currency and therefore cannot at the same time use it to control other local economic conditions \cite{Fleming1962, Mundell1963}. Although there is some alignment between business cycles in Cambodia, the currency peg introduces a source of exogenous variation in Cambodian monetary policy that we will exploit in the empirical analysis. In general, the NBC has set the official exchange rate with the difference of one percent as compared to the market rate. The exchange rate Khmer riel against US dollar was set aound 4,000 riels per \$1. The exchange rate of Vietnam dong against Khmer riel and Thai baht against Khmer riel are associated with bilateral exchange rate of US dollar against Vietnam dong and US dollar against Thai baht. 


Local currency is estimated as just 10 percent of the total board money. In \citet{Khou2013} showed that circulation of the riel has grown since the early 2000s, in particular, in the second half. Between 1989 and 2000, the amount of riels in circulation varied slightly and was around 100 million dollars, while the economic growth rate exceeded 5 percent on average. In contrast, it increased by about 20 percent between 2001 and 2010. This allowed the volume of the banknotes in circulation to pass from about 100 million dollars in the period of 1989-2000 to nearly 800 million dollars in 2010. Yet bank deposits in riel remain very low; they represent less than 5 percent of the total deposits. 


\\

The reserve requirement is the main Cambodia's MP to control the prudential and liquidity management. As the time of the pandemic, many economic activities was hard hit in demand and supply shock, to facilitate sable money supply in the market, the MP Committee decuted  to reduce the reserve requirement for both local currency and foreign currency US dollar to 7 percent.\footnote{The decision to counties the rate of reserve requirement of 7 percent was introduced since 2021. The summary result of the 56 Monetary Policy Committee Meeting was released in the official Facebook page of NBC on 23 August 2021. The settlement can be found via this link: \url{https://www.facebook.com/nationalbankofcambodiaofficial/photos/3897621040348119}} 

\\


\\



%Much of the attempt from  central bank of Cambodia is to provide the stability macroeconomics policy, in particular the stable monetary policy, within the US and Riel exchange rate around 4000 Riels per \$1 and the inflation rate around 3 percent per year. 

%-----------------------------------------------------------------------------------
\subsection{Household Income}
%-----------------------------------------------------------------------------------
In 2020, the total number of Cambodian normal households is about 3.6 million with 15.9 million people living in Cambodia \cite{NIS2020}. Gross Domestic Product (GDP) per Capita has increased from \$1,555 in 2018 to \$1,694 in 2019. 

The total monthly income of Cambodian households is estimated to be \$567 in 2019/20, which is an increase by 16 percent if compared to 2017 and it is increased by 58 percent if compared to year 2014. Cambodian household’s disposable income in 2019/20 increased by 16 percent if compared to 2017 and increased by 59 percent if compared to 2014. The higher increase compared to the increase of total income is mainly due to the data collection on current transfers paid changed from diary methods to recall method since 2012.


The central bank was first introduced an 18 percent cap on annual interest rate of microfinance institute loans both denominated in local and US dollar currency. Even through not much prevent literature to study the inference of this interest rate ceiling on household. But we can assume that it will direct benefit to family financing.\footnote{For example, in \citet{Heng2021}, they studied the effect of the interest rate cap of MFIs and financial inclusion by using the dataset on financial institution and aggregate level from 2016-2019. They result found}  

Since the end of the financial crisis in 2009, currently the community transmission of COVID-19 has been a major challenge for Cambodia's healthcare system and household incomes. According to the \cite{Karamba2021}, a fifth-round result of the World Bank's High-Frequency Phone Survey of Households in Cambodia found that 45 percent of households experienced declines in income in December 2020 and March 2021. This compares to 48 percent in Round 4 that experienced declines between October and December, 51 percent in Round 3 that experienced declines between August and October. Where in Round 2, households income report declines 63 percent between May and August 2020 and 83 percent in Round 1 that experienced declines between the first COVID-19 outbreak and May 2020. Despite the fact that Phnom Penh and its nearby town were put in lockdown, but we does not yet any report to show about household income lost during the period of lockdown and travail restrictions.  
%-----------------------------------------------------------------------------------
Many decades after the fall of the Khmer Rouge, now Cambodia has moved to new transposition with a lower middle-income status since 2015. 
\clearpage
%-----------------------------------------------------------------------------------
\section{A Model of Monetary Policy Transmission Channel}\label{sec:model}
%-----------------------------------------------------------------------------------
%-----------------------------------------------------------------------------------

\subsection{Monetary Policy Transmission}
%-----------------------------------------------------------------------------------
\subsubsection{Traditional Monetary Policy}
Theoretically, the central bank controls the nominal interest rate it on nominal bonds. Perfect foresight implies that the real interest rate is:

%-----------------------------------------------------------------------------------
\begin{equation}
	1 + r_{t} = \frac{1 - i_{t}}{1+ \pi_{t+1}}
\end{equation}
%-----------------------------------------------------------------------------------
where $\pi_{t+1}$ is the rate of price inflation, $1 + \pi_{t+1} \equiv P_{t +1}$ . We consider three specifications of $P_{t}$ monetary policy. Under neoclassical policy, the central bank sets a path for it that is consistent with $L_{t} = 1$ for all $t$. In doing so, it achieves a path $r_{t}^{\ast}$ for the real interest rate, and the economy behaves as if the wage rigidity constraint (8) was absent. We follow the literature and call $r_{r}^{\ast}$ the ‘natural interest rate’ path. Under constant-r policy, the central bank targets a real rate that is constant at the economy’s steady-state natural rate, $r_{t} = r^{\ast}$, and does not change this target in response to any of our experiments. This policy shuts off all equilibriating real interest rate movements, including those driven by changes in expected inflation. Finally, our benchmark monetary policy is one in which the monetary authority sets the nominal interest rate according to a Taylor rule subject to the zero lower bound:
%-----------------------------------------------------------------------------------
\begin{equation}\label{q2}
	i_{t} = \max \quad \left(0, (1 + r^{\ast})(1 + \pi^{\ast}) \left( \frac{\frac{P_{t}}{P_{t -1}}}{1 + \pi^{\ast}}\right)^{\phi} -1 \right)
\end{equation}
%-----------------------------------------------------------------------------------
where $r^{\ast}$ is the steady state natural rate, π∗ is the inflation target, and $\phi > 1$. In our calibrated exercises, the zero lower bound in (\ref{q2}) will always be binding.
%-----------------------------------------------------------------------------------

\subsubsection{The Identification of Romer and Romer on Monetary Policy Shocks}
To analysis the influence  of MP on inequality in Cambodia, we follow \citet{Romer2004} to identify innovations to monetary policy purged of anticipatory effects associated with economic conditions. They construct a historical measure of changes in the target the US's Federal Fund rate at each FOMC meeting from the period of 1969 to 1996 by using the real-time forecasts of the Fed staff presented in the Greenbook prior to each FOMC meeting. Conversely, the context of MP in Cambodia, it has their own unique with much as much different from MP in the US. Cambodia used multiples foreign currencies and the US dollar is the main foreign currency in terms of the mean payment, a unit of account and a store of value. Interestingly, the US dollar has a huge share in the market concentration with about 90 percent. The Cambodian MP probably depend on the Fed.

To the best of our knowledge, we develop a new monetary policy shocks for applied in the context of Cambodia based on \citet{Romer2004}. We assume MP in the Kingdom has two main component: to control monetary supply in the Cambodian riel and the US dollar. The model has the nurture below: 
%-----------------------------------------------------------------------------------
\begin{equation}\begin{split}\label{q3}
		\Delta rrr_{m} = \alpha + \beta rrrb_{m} + \sum_{i = -1}^{2}{\gamma_{i} \widetilde{\Delta y}_{m,i}} + \sum_{i = -1}^{2} \lambda_{i} (\widetilde{\Delta y}_{m,i} - \widetilde{\Delta y}_{m-1, i}) + \sum_{i = -1}^{2} \varphi_{i}\widetilde{\pi}_{m,i} \\
		+ \sum_{i = -1}^{2}\theta_{i}(\widetilde\pi_{m,i} - \widetilde\pi_{m-1, i}) + \rho \widetilde{u}_{m,0} + \varepsilon_{m}
\end{split}\end{equation}
%-----------------------------------------------------------------------------------
 where $m$ dotes the MP Committee meeting, $\Delta rrr_{m}$ is the change in the intended reserve requirement rate of the central bank around the innovation policy with the new implementation $m$ for the Cambodian riel currency. $rrrb_{m}$ is the level of the intended interest rate before any changes associated with any new regulations $m$ for the Cambodian riel. $\widetilde{\pi}$, $\widetilde{\Delta y}$, and $\widetilde{u}$ refer to the forecasts of inflation, real out growth, and the unemployment rate. The $i$ subscripts refer to the horizon of the forecast: $-1$ is the previous quarter, $0$ is the current quarter, while $1$ and $2$ are one and two quarters ahead, respectively. The forecast for the previous quarter is often actual data rather than a forecast.  
 
 Similarly, as equation (\ref{q3}), MP shocks for the US dollar supply in Cambodia's market has a structure as follow: 
%-----------------------------------------------------------------------------------
\begin{equation}\begin{split}\label{q4}
		\Delta rrd_{m} = \alpha + \beta rrdb_{m} + \sum_{i = -1}^{2}{\gamma_{i} \widetilde{\Delta y}_{m,i}} + \sum_{i = -1}^{2} \lambda_{i} (\widetilde{\Delta y}_{m,i} - \widetilde{\Delta y}_{m-1, i}) + \sum_{i = -1}^{2} \varphi_{i}\widetilde{\pi}_{m,i} \\
		+ \sum_{i = -1}^{2}\theta_{i}(\widetilde\pi_{m,i} - \widetilde\pi_{m-1, i}) + \rho \widetilde{u}_{m,0} + \varepsilon_{m}
\end{split}\end{equation}
%-----------------------------------------------------------------------------------
Everything are the same, but the change is the intended reserve requirement rate of the central bank around MP Committee $m$ for the US dollar denote $\Delta rrd_{m}$, and the level of the intented interest rate before any change associated with meeting $m$ has a notation $rrdb_{m}$.  


%------------------------------------------------------------------
\subsection{Monetary}
\subsubsection{Household Heterogeneity and Aggregate Consumption}

We consider a household with separable preference over nondurable consumption $c_{t}$ and hour of work $n_{t}$. We assume no aggregate uncertainty for simplicity: the same insights obtain when markets are compete, except with respect to idiosyncratic shock to their income and spending needs. The household is endowned with a stream of real unearned income $y_{t}$. He has perfect foresight over the general level of price $P_{t}$ and the path of the nominal wages $W_{t}$, and holds long-term nominal and real contracts. The time horizon finite or infinite with discrete period $t= 0, 1, 2, \ldots$ and the agent solves the following utility maximization function:  

%-----------------------------------------------------------------------------------
\begin{equation*}
	\max \quad \mathbb{E} \left[\sum_{t=0}^{\infty}\beta^{t}\begin{Bmatrix}u (c_{t}) - v(n_{t})\end{Bmatrix}\right]
\end{equation*}
%-----------------------------------------------------------------------------------
%-----------------------------------------------------------------------------------
\begin{equation}\begin{split}\label{e1}
	\text{s.t.} \quad \quad P_{t}c_{t} = P_{t}y_{t} + W_{t}n_{t} + (_{t-1}B_{t}) + \sum_{s \geqslant 1}(_{t}Q_{t+s})(_{t-1}B_{t+s} - _{t}B_{t+s}) + P_{t}(_{t-1}b_{t}) \\
	+ \sum_{s \geqslant 1}(_{t}q_{t+s}) P_{t+s} (_{t-1}b_{t+s} - _{t}b_{t+s})
\end{split}\end{equation}
%-----------------------------------------------------------------------------------

The follow budget constraint in equation (\ref{e1}) view the consumer in every period $t$, as having a portfolio of zero coupon bonds inherited from period $t-1$ and a portfolio of bonds to carry into the next period.\footnote{Of course, just decide to roll over household position from the previous period. This corresponds to the costles trade that sets $_{t-1}b_{t+s} = _{t}b_{t+s}$ and $_{t}B_{t+s} = _{t-1}B_{t+s}$ for all $s$} $_tQ_{t+s}$ denote the time $t$ price of a nominal zero coupon bond paying at time $t+s$, $_{t}q_{t+s}$ is the price of a real zero coupon bond, and $_{t}B_{t+s}$ denote the nominal quantities purchased and $_{t}b_{t+s}$ is the real quantities purchased by households in each period, respectively. To keep the problem well-defined, we assume that the prices of nominal and real bonds prevent arbitrage profits. This implied a Fischer equation for the nominal term structure: 
%-----------------------------------------------------------------------------------
\begin{equation*}
	_{t}Q_{t+s} = (_{t}q_{t+s}) \frac{P_{t}}{P_{t+s}} \quad \quad \quad \forall t,s
\end{equation*}
%-----------------------------------------------------------------------------------
As we focus on the period $t=0$. Now we represent stocks which pay a riskless real dividend stream and therefore are priced associated with the risk-free discounted value of this stream, inflation-indexed government bonds, and price-level adjusted mortgages. We write the real wage at $t$ as $w_{t} \equiv \frac{W_{t}}{P_{t}}$, the initial real term structure as $q_{t} \equiv {_{0}q_{t}}$, the initial nominal term structure as $Q_{t} \equiv {_{0}Q_{t}}$, and impose the present-value normalization $q_{0} = Q_{0} =1$.

Using either a terminal condition, the flow budget constraints consolidate into an intertemporal budget constraint: 
%-----------------------------------------------------------------------------------
\begin{equation}\label{e2}
	\sum_{s \geqslant 0}q_{t}c_{t} = \underbrace{\sum_{s \geqslant 0}q_{t}(y_{t} + w_{t}n_{t})}_{\omega^H} + \underbrace{\sum_{s \geqslant 0}q_{t}\left ( (_{1}b_{t}) + \left(\frac{_{1}B_{t}}{P_{t}}\right)\right )}_{\omega^F} \equiv \omega 
\end{equation}
%-----------------------------------------------------------------------------------
The present value of consumption must be equal to wealth $\omega$: the sum of human wealth $\omega^{H}$ equal to the present value of all future income, and financial wealth $\omega^{F}$. Since $_{-1}B_{t}$ and $_{-1}b_{t}$ only enter to equation (\ref{e2}) through $\omega_{F}$, it follows that financial assets with the same initial present value deliver the same solution to the consumer problem. For instance, this framework predicts that a household with an adjustable-rate mortgage (ARM), with $_{-1}B_{0} = -L$, chooses the same plan for consumption and labour supply as an otherwise identical household with a fixed-rage mortgage (FRM), $_{-1}B_{t} = -M$ for $t = 0, 1, 2, \ldots, T$, provided the two mortgages have the same outstanding principal, $L = \sum_{t=0}^{T}Q_{t}M$. In this sense, the composition of balance sheets is irrelevant. 
%-----------------------------------------------------------------------------------

\subsubsection{Adjustment After a Transitory Shock}
The first-order change in initial consumption $dc \equiv dc_{0}$, labor supply $dn \equiv dn_{0}$, and welfare $dU$ that results from this change in the environment. 
%-----------------------------------------------------------------------------------
\begin{align}
	\label{e3} dc &= MPC(d\Omega + \psi ndw) - \sigma c MPS \frac{dR}{R} \\ 
	\label{e4} dn &= MPN(d\Omega + \psi ndw) + \psi n MPS \frac{dR}{R} + \psi n \frac{dw}{w} \\ 
    \label{e5} dU &= u' (c) d \Omega 
\end{align}
%-----------------------------------------------------------------------------------
Let $\sigma$ and $\psi$ be the local Frisch elasticities of substitution in consumption and hours. The traditional theory denie the marginal propensity to consume as $MPC =\frac{\partial c_{0}}{\partial y_{0}}$ along the initial path. When a consumer exogenously receives an extra dollar of income, he probaly increase consumption by $MPC$ dollars, but to the extent that labor supply is elastic $\psi <0$, he also reduces hours by marginal propensity to nominal $MPN = \frac{\partial n_{0}}{\partial y_{0}} <0$, leaving only marginal propensity to saving $MPS = 1 - MPC + w_{0}MPN$ dollars for saving. Indeed, the behavioral response to income changes turn out to also matter for the response to the real interest rate, wage, and price level changes. At the same time, the net of consumption wealth change $d\Omega$ has the structure below: 
%-----------------------------------------------------------------------------------
\begin{equation}\label{e6}
	d\Omega = dy +ndw - \sum_{s \geqslant 0}Q_{t}\left(\frac{_{-1}B_{t}}{P_{0}}\right) \frac{dP}{P} + \left(y + wn + \left(\frac{_{-1}B_{0}}{P_{0}}\right) + (_{-1}b_{0}) - c\right) \frac{dR}{R}
\end{equation}
%-----------------------------------------------------------------------------------
As follows from an application of Stutsky's equations, the relative price change $dR$ and $dw$ generate substitution effects on consumption and labor supply with familiar signs, and magnitudes given by a combination of Frisch elasticities and marginal propensities. 
%-----------------------------------------------------------------------------------
\subsubsection{The Net Wealth Revaluation}
As shows in equation (\ref{e6}), the first term, $dy + ndw$ is the traditional effect from the change in the present value of income. This is the sum of the unearned income gain $dy$, and the change in earned income holding hours fixed $ndw$. The second term represents the effect from the immediate and permanent increase of assets and liabilities. Define the household's net nominal position $NNP$ as the present value of his nominal assets: 
%-----------------------------------------------------------------------------------
\begin{equation*}
	NNP \equiv \sum_{s \geqslant 0} Q_{t} \left(\frac{_{-1}B_{t}}{P_{0}}\right)
\end{equation*}
%-----------------------------------------------------------------------------------
Suppose for example that nominal prices unexpectedly rise inflation by $\frac{dP}{P} = 1\%$. A nominal saver with $NNP = \$30k$ experiences a wealth effect of $-NNP\frac{dP}{P}$, thus looses the equivalent of \$300. Conversely, a nominal borrower with $NNP = -\$30k$ gains the equivalent of \$300. 

The final term in $d\Omega$ is the wealth effect from the change in the real interest rate. If we define the household's inhedged interest rate exposure as: 
%-----------------------------------------------------------------------------------
\begin{equation*}
	URE \equiv y + wn + \left(\frac{_{-1}B_{0}}{P_{0}}\right) + (_{-1}b_{0}) -c 
\end{equation*}
%-----------------------------------------------------------------------------------
This term is equal to $URE\frac{dR}{R}$. It represents the net saving requirement of the household at time $0$, from the point of view of date $-1$. It includes the stocks of financial assets that mature at date $0$ rather than interest flows. But the $URE$ is the difference between all maturing assets (including income) and liabilities (including planned consumption) at time $0$.  
%-----------------------------------------------------------------------------------
\iffalse 
\begin{equation*}
	\sum_{t \geqslant 1}q_{t}(y_{t} + w_{t}n_{t}) + \sum_{t \geqslant 1}q_{t}\left((_{-1}b_{t}) + \left(\frac{_{-1}B_{t}}{P_{t}}\right)\right) - \sum_{t \geqslant 1}q_{t}c_{t} = -URE
\end{equation*}
}
\fi
%-----------------------------------------------------------------------------------

Since equation (\ref{e3})--(\ref{e5}) draws a distinction between exogenous changes in income and changes in wages, the next have rewrite the consumption response as a function of the total income change, inclusive of the labor supply response. Given an overall change in income $dY = dy + ndw + wdn$, the household's consumption response is: 
%-----------------------------------------------------------------------------------


\begin{equation}\label{e7}
	dc = M\widehat{P}C \left ( dY - NNP \frac{dP}{P} + URE \frac{dR}{R} \right) - \sigma c (1- M\widehat{P}C)\frac{dR}{R} 
\end{equation}
\vspace{-.7em}
\begin{equation*}
 \text{where} \quad \quad M\widehat{P}C = \frac{MPC}{MPC + MPS} = \frac{MPC}{1 + \omega MPN} \geqslant MPC
\end{equation*}
%-----------------------------------------------------------------------------------
Hence, once we have factored in the endogenous response of income to transfers, the relevant marginal propensity to consume be $M\widehat{P}C$, the number has run between $0$ and $1$ that determines how the remaining amount of income is split between consumption and savings. 
%-----------------------------------------------------------------------------------
\subsubsection{Shocks Under Incomplete Markets}
As the previous section, it will only discuss influence behavior through its impact on the $MPC$. To model market incompleteness in a general form, we assume that the consumer can trade in $N$ stocks as well as in a nominal long-term bond. In period $t$, stocks pay real dividends $d_{t} = (d_{1t}, \ldots, d_{Nt})$ and can be purchased at real prices $S_{t} = (S_{1t}, \ldots, S_{Nt} )$; the consumer's portfolio of shares is denoted by $\theta_{t}$. Following the standard formulation in the macroeconomic textbook, we assume that the long-term bond can be bought at time $t$ at price $Q_{t}$ and is a promise to pay a geometrically declining nominal coupon with pattern $(1, \delta, \delta^{2}, \ldots)$ starting at date $t+1$. The current coupon which we denote $\Lambda_{t}$, then summarizes the entire bond portfolio, thus it is not necessary to separately keep track of future coupons. The household's budget constraint at date $t$ is now: 
%-----------------------------------------------------------------------------------
\begin{equation}
	P_{t}c_{t} + Q_{t}(\Lambda_{t+1} - \delta \Lambda_{t}) + \theta_{t+1} \cdot P_{t}S_{t} = P_{t}y_{t} + P_{t}w_{t}n_{t} + \Lambda_{t} + \theta_{t} \cdot (P_{t}S_{t} + P_{t}d_{t})
\end{equation}
%-----------------------------------------------------------------------------------
A borrowing constraint limits trading. This constraint specifies that real end-of-period wealth cannot be too negative: specifically,      
%-----------------------------------------------------------------------------------
\begin{equation}\label{e9}
	\frac{Q_{t}\Lambda_{t+1} + \theta_{t+1} \cdot P_{t}S_{t}}{P_{t}} = - \frac{\overline{D}}{R_{t}}
\end{equation}
%-----------------------------------------------------------------------------------
for some $\overline{D} \geq 0$, where $R_{t}$ is the real interest rate at time $t$. The constraint in equation (\ref{e9}) is a standard specification for borrowing limits. 

Provided that the portfolio choice problem just described has a unique solution at date $t-1$, the household's net nominal position and his unhedged interest rate exposure are both uniquely pinned down in each state at time $t$. The consumer was indifferent between all portfolio choices. Here, these quantities are defined as: 
%-----------------------------------------------------------------------------------
\begin{align*}
    NNP_{t} &\equiv (1 + Q_{t}\delta) \frac{\Lambda_{t}}{P_{t}} \\
    URE_{t} &\equiv y_{t} + w_{t}n_{t} + \frac{\Lambda_{t}}{P{_t}} + \theta_{t} \cdot d_{t} - c_{t}
\end{align*}
%-----------------------------------------------------------------------------------
As before, $NNP_{t}$ is the real market value of nominal wealth: the sum of the current coupon, $\Lambda_{t}$, and the value of the bond portfolio if it were sold immediately, $Q_{t}\delta\Lambda_{t}$. Similarly, $URE_{t}$ is maturing assets (including income, real coupon payments and dividends) net of maturing liabilities (including consumption). 

Consider the predicted effects on consumption resulting from a simultaneous unexpected change in his current unearned income $dy$, his current real wage $dw$, the general price level $dP$ and the real interest rate $dR$, for one period only. Assume that this variation leads asset prices to adjust to reflect the change in discounting alone: $\frac{dQ}{Q} = \frac{dS_{j}}{S_{j}} = - \frac{dR}{R}$ for $j = 1, \ldots, N$. If $MPC = \frac{\partial c}{\partial y}$, and both $MPN$ and $MPS$ are similarly defined as the responses to current income transfers, them the positive results from equation (\ref{e3})--(\ref{e5}) carry through.   
%-----------------------------------------------------------------------------------

Assume that the consumer is at an interior optimum, at a binding borrowing constraint, or unable to access financial markets. Let $MPS = 0$, than his first order change in consumption $dc$ and labor supply $dn$ continue to be given by equations (\ref{e3}) and (\ref{e4}). In particular, writing $M\widehat{P}C \equiv \frac{MPC}{MPC + MPS}$, the relationship between $dc$ and the total change in income $dY = dy + ndw + wdn$ is still given by equation (\ref{e7}). 
%-----------------------------------------------------------------------------------

\subsubsection{Monetary Redistribution Channel}
%-----------------------------------------------------------------------------------
Aggregation of consumer responses as described by equation (\ref{e7}) shows that the per capita aggregate consumption change can be decomposed as the sum of five channels. To first order, in response to $dY_{i}$, $dY$, $dP$ and $dR$, aggregate consumption changes by: 
%-----------------------------------------------------------------------------------
\begin{equation}\begin{split}\label{e10}
		dC = \underbrace{\mathbb{E}_{I} \left[\frac{Y_{i}}{Y} M\widehat{P}C_{i}\right] dY}_{\text{Aggregate income channel}} + \underbrace{\mathrm{Cov}_{I} \left(M\widehat{P}C_{i}, dY_{i} - Y_{i} \frac{dY}{Y}\right)}_{\text{Aggregate income channel}} - \underbrace{\mathrm{Cov}_{I} (M\widehat{P}C_{i}, NNP_{i}) \frac{dP}{P}}_{Fisher channel} \\
		+ \left( \underbrace{\mathrm{Cov}_{I} (M\widehat{P}C_{i}, URE_{i})}_{\text{Interest rate exposure channel}} - \underbrace{\mathbb{E}_{I} \left[ \omega_{i} (1 - M\widehat{P}C_{i})c_{i}\right]}_{\text{Substitution channel}}\right) \frac{dR}{R}
\end{split}\end{equation}
%-----------------------------------------------------------------------------------
The proof is given in appendix A.1. Decomposing $i$'s individual income changes as $dY_{i} = \frac{Y_{i}}{Y}dY + dY_{i} - \frac{Y_{i}}{Y}dY$ the sum of an aggregate component and a redistributive component, and using market clearing condition, the fiscal rule, and the fact that $\mathbb{E}_{I}\left[dY_{i} - \frac{Y_{i}}{Y} dY\right] = 0$ to transform expectations of products into covariances. This equation holds irrespective of the underlying model generating $MPCs$ and exposures at the micro level, as well as the relationship between $dY$, $dP$, and $dR$ at the macro level. Most of the bracketed terms are cross-sectional moments that are measurable in household level micro-data and are informative about the economy's macroeconomic response to a shock, no matter the source of this shock. 
%-----------------------------------------------------------------------------------

Heterogeneity implies a role for redistribute channels in the monetary transmission mechanism, except under special conditions. For intence, if aggragate income is distribted proportionlly to individual income, thus that $dY_{i} = \frac{Y_{i}}{Y}dY$; if no equilibrium asset trade is possible, so that agents consume all their incomes $Y_{i} = c_{i}$ and $NNP_{i} = URE_{i} = 0$; and if all agents have the same elasticity of intertemporal substitution $\sigma_{i} = \sigma$, then the representative agent response $\frac{dC}{C} = \sigma\frac{dR}{R}$ obtains even under heterogeneity. 

%-----------------------------------------------------------------------------------
The redistributive channels of MP can be signed and quantified by measuring the covariance terms in equation (\ref{e10}), either directly in micro data. The data suggests that following covariance is true:  
%-----------------------------------------------------------------------------------
\begin{align}
	\label{e11} \mathrm{Cov}_{I} (M\widehat{P}C_{i}, URE_{i}) &< 0 \\
	\label{e12} \mathrm{Cov}_{I} (M\widehat{P}C_{i}, NNP_{i}) &< 0 \\
	\label{e13} \mathrm{Cov}_{I} (M\widehat{P}C_{i}, Y_{i}) &< 0
\end{align}

%-----------------------------------------------------------------------------------


%-----------------------------------------------------------------------------------
\iffalse 
\begin{equation}
	\gamma_{i} \equiv \frac{\partial \left(\frac{Y_{i}}{Y} - 1 \right) Y}{\left( \frac{Y_{i}}{Y} - 1 \right)} \frac{Y}{\partial Y} 
\end{equation}
\fi
%-----------------------------------------------------------------------------------

%-----------------------------------------------------------------------------------
\begin{equation}\label{e18}
	\frac{dC}{C} = ( \mathcal{M} + \gamma \mathcal{E}_{Y}) \frac{dY}{Y} - \mathcal{E}_{P} \frac{dP}{P} + (\mathcal{E}_{R} - \sigma S) \frac{dR}{R} 
\end{equation}
%-----------------------------------------------------------------------------------
% Please add the following required packages to your document preamble:

% \usepackage{graphicx}
\begin{table}[]
	\centering
	\caption{The Cross-Sectional Moments that Determine Consumption in Equation (\ref{e18})}
	\label{t1}
	\resizebox{.95\textwidth}{!}{%
		\begin{tabular}{@{}cccc@{}} 
			\toprule
			\textbf{Variable} & \textbf{Definition} & \textbf{Description} & \textbf{MP Channel} \\ \midrule
			$\mathcal{E}_{R}$ & $\mathrm{Cov}_{I} \left(MPC_{i}, \frac{URE_{i}}{\mathbb{E}_{I} \left[c_{i}\right]}\right)$ & Redistribution elasticity for $R$ & Interest rate exposure \\
			
			$\mathcal{E}_{R}^{NR}$ & $\mathbb{E}_{I} \left[ MPC_{i} \frac{c_{i}}{\mathbb{E}_{I} \left[c_{i}\right]}\right]$ & No rebate & --- \\
			
			$\widehat{S}$ & $\mathbb{E}_{I} \left[(1 - MPC_{i})\frac{c_{i}}{\mathbb{E}_{I}\left[c_{i}\right]}\right]$ & Hicksian scaling factor & Substitution \\ 
			\midrule
			
			$\mathcal{E}_{P}$ & $\mathrm{Cov}_{I} \left(MPC_{i}, \frac{NNP_{i}}{\mathbb{E}_{I}\left[c_{i}\right]}\right)$ & Redistribution elasticity for $P$ & Fisher\\
			
			$\mathcal{E}_{P}^{NR}$ & $\mathbb{E}_{i}\left[MPC_{i}\frac{NNP_{i}}{\mathbb{E}_{I}\left[c_{i}\right]}\right]$& No rebate & --- \\
			\midrule
			$\mathcal{E}_{Y}$ & $\mathrm{Cov}\left(MPC_{i}, \frac{Y_{i}}{\mathbb{E}_{I}\left[c_{i}\right]}\right)$ & Redistribution elasticity for $Y$ & Earnings heterogeneity \\
			
			$\mathcal{M}$ & $\mathbb{E} \left[MPC_{i}\frac{Y_{i}}{\mathbb{E}_{I}\left[c_{i}\right]}\right]$ & Income-weighed $MPC$ & Aggregate income \\
			\bottomrule
		\end{tabular}%
	}
\begin{tablenotes}
	\small
	\item \textbf{Note:} This table based on the proof in appendix A2 and according to \citet{Auclert2019}. 
\end{tablenotes} 

\end{table}

%-----------------------------------------------------------------------------------
\begin{equation}
	URE_{i} = Y_{i} - T_{t} - C_{i} + A_{i} - L_{t}
\end{equation}
%-----------------------------------------------------------------------------------

%-----------------------------------------------------------------------------------
\begin{equation}
	\mathrm{Cov}(MPC_{i}, URE_{i}) = \underbrace{\mathrm{Cov}(\mathbb{E} \left[MPC_{i} | Z_{i}\right], \mathbb{E} \left[URE_{i} | Z_{i}\right])}_{\text{Explained fraction of covariance}} + \underbrace{\mathbb{E}\left[\mathrm{Cov} (MPC_{i}, URE_{i} | Z_{i})\right]}_{\text{Unexplained fraction of covariance}}
\end{equation}
%-----------------------------------------------------------------------------------

%-----------------------------------------------------------------------------------
\begin{align*}
	MPC_{i} &= \alpha_{M} + \beta_{M}Z_{i} + \epsilon_{Mi} \\
	URE_{i} &= \alpha_{R} + \beta_{R}Z_{i} + \epsilon_{Ri}
\end{align*}
%-----------------------------------------------------------------------------------

%-----------------------------------------------------------------------------------
\begin{equation}
\begin{align*}
	\mathrm{Cov} (MPC_{i}, URE_{i}) &= \mathrm{Cov}\left(\widehat{MPC_{i}} + \widehat{\epsilon_{Mi}}, \widehat{URE_{I}} + \widehat{\epsilon_{Ri}}\right) \\
	&= \mathrm{Cov}\left(\widehat{\beta_{M}}Z_{i} + \widehat{\epsilon_{Mi}}, \widehat{\beta_{R}}Z_{i} + \widehat{\epsilon_{Ri}}\right) \\
	&= \mathrm{Var} \left(Z_{i}\right)\widehat{\beta_{M}}\widehat{\beta_{R}} + \mathrm{Cov} \left(\widehat{\epsilon_{Mi}}, \widehat{\epsilon_{Ri}}\right)
\end{align*}
\end{equation}
%-----------------------------------------------------------------------------------

%-----------------------------------------------------------------------------------
%-----------------------------------------------------------------------------------

%-----------------------------------------------------------------------------------
%-----------------------------------------------------------------------------------




%------------------------------------------------------------------
%------------------------------------------------------------------
\section{Data Sources and Descriptive Statistics}\label{sec:data}
%-----------------------------------------------------------------
\subsection{Sources, Variables and Sample}	
Our analysis focus sorely Cambodia over the period 2008Q1-2020Q4. In fact, surveys have been used by several studies on the im- pact of monetary policy on income inequality (e.g., Coibion et al. (2017) for the U.S.; Mumtaz and Theophilopoulou (2017) for the U.K.; Inui et al. (2017) for Japan; and Casiraghi et al. (2018) for Italy).
%-----------------------------------------------------------------------------------
\subsection{Percentiles and the Inequality Index}	
%-----------------------------------------------------------------------------------

%-----------------------------------------------------------------------------------
\section{Empirical Results}\label{sec:results}
%-----------------------------------------------------------------------------------

%-----------------------------------------------------------------------------------
\subsection{The Effects of Monetary Policy Shocks}
%-----------------------------------------------------------------------------------

%-----------------------------------------------------------------------------------
\subsection{Redistribution Elasticities of Household Economy}
%-----------------------------------------------------------------------------------

%-----------------------------------------------------------------------------------

\section{Robustness Checks}\label{sec:robust}

%-------------------------------------------------------------------

%-----------------------------------------------------------------------------------
\section{Discussion}
%-----------------------------------------------------------------------------------
  
\section{Conclusion}\label{sec:conclusion}
%-----------------------------------------------------------------------------------

%-------------------------------------------------------------------
\clearpage	
{\small
	%\bibliographystyle{jpe} % is refer to the Journal of Political Economy
	%\bibliographystyle{apecon}
	%\bibliographystyle{ier}
	\bibliographystyle{aea}  % is refer to the American Economic Association
	%\bibliographystyle{ecca} % is refer to the Journal of Economica
	\bibliography{References}
}	
%--------------------------------------------------------------------
\clearpage

\appendix 
	\part*{Appendix}\label{app:dix}
	\pagenumbering{Arabic}
	\renewcommand*{\thepage}{A\arabic{page}}
	\numberwithin{equation}{section}
	\renewcommand{\thetable}{A\arabic{table}}
	%\renewcommand{\thefigure}{A\arabic{figure}}	
	
\end{document}
